%!TEX root = tcc.tex
\chapter{Introdução} \label{ch:introducao}

\lipsum[1]

\lipsum[2]

\lipsum[3]

\section{Apresentação do problema}

\lipsum[1]

\section{Objetivos} \label{sec:obj}

\lipsum[1]

\subsection{Geral} \label{subsec:obj-general}

\lipsum[1]

\subsection{Específicos} \label{subsec:obj-spec}

\begin{enumerate}
  \item Pellentesque habitant morbi tristique senectus
  \item Phasellus eu tellus sit amet tortor gravida placerat
\end{enumerate}

\section{Justificativa}

\lipsum[1]

\section{Estrutura do trabalho}

Esse trabalho está dividido em 5 capítulos. O \cref{ch:introducao} traz
aspectos introdutórios, para compreender problemas relacionados e os objetivos, seguindo para uma solução que será detalhada nos capítulos posteriores.

O \cref{ch:rev-biliografica} apresenta os conceitos, técnicas utilizadas no desenvolvimento do trabalho.

O \cref{ch:solucao} apresenta a solução desenvolvida.

O \cref{ch:aplicacao} apresenta os resultados coletados com a aplidação da solução.

No \cref{ch:conclusao} é apresentada a conclusão do trabalho baseando-se nos resultados conquistados ao longo da pesquisa e desenvolvimento.

