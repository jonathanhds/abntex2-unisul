\documentclass{abntex2-unisul}

\usepackage{lipsum} % para geração de dummy text

\titulo{<O Título do Seu Trabalho Aqui>}
\autor{<Nome Aluno 1>\\ <Nome Aluno 2>}
\local{<Cidade>}
\data{<Ano de Publicação>}

\orientador[Orientador:]{<Nome do(a) Orientador(a)>}
\coorientador[Coorientador:]{<Nome do(a) Coorientador(a)>}

\instituicao{Universidade do Sul de Santa Catarina}
\tipotrabalho{Trabalho de Conclusão de Curso}

\naturezatrabalho{Este Trabalho de Conclusão de Curso foi julgado adequado à obtenção do título de... <Bacharel, Licenciado, Mestre, Especialização ou Tecnólogo> em... <área de concentração> e aprovado em sua forma final pelo Curso de... <nome do curso>, da \imprimirinstituicao.}
\preambulo{Trabalho de Conclusão de Curso apresentado ao Curso de <graduação, pós-graduação ou tecnólogo> em... <nome do curso>, da \imprimirinstituicao, como requisito parcial para obtenção do título de.. <Bacharel, Licenciado, Mestre, Especialização ou Tecnólogo>.}

\membrobancaA[\imprimirinstituicao]{Prof. e orientador \imprimirorientador}
\membrobancaB[Universidade B]{Membro B}
\membrobancaC[Universidade C]{Membro C}

\begin{document}

    % Elementos pré-textuais
    \pretextual
    \imprimircapa
    \imprimirfolhaderosto{}
    %!TEX root = tcc.tex

\begin{errata}
Errata!
\end{errata}

    \imprimirfolhadeaprovacao
    %!TEX root = tcc.tex

\begin{dedicatoria}
Texto das dedicatórias. Texto das dedicatórias. Texto das dedicatórias. Texto das dedicatórias. Texto das dedicatórias. Texto das dedicatórias. Texto das dedicatórias. Texto das dedicatórias. Texto das dedicatórias.
\end{dedicatoria}
    %!TEX root = tcc.tex

\begin{agradecimentos}

\lipsum[1]

\end{agradecimentos}
    %!TEX root = tcc.tex

\begin{epigrafe}
"Epígrafe epígrafe epígrafe epígrafe epígrafe epígrafe epígrafe epígrafe epígrafe epígrafe epígrafe epígrafe epígrafe epígrafe epígrafe." (AUTOR, ano)
\end{epigrafe}
    \begin{resumo}

Lorem ipsum dolor sit amet, consectetur adipiscing elit. Suspendisse adipiscing eu libero non ultricies. Donec accumsan, turpis ut malesuada facilisis, felis augue porta quam, et lacinia turpis ligula et risus. Suspendisse lobortis ipsum ligula, ut accumsan tortor tempus lacinia. Praesent eu nisi vehicula, cursus ipsum vel, suscipit nulla. Mauris fringilla ullamcorper orci eu ullamcorper. Praesent eget tristique magna. Nulla varius libero placerat, viverra nisi volutpat, iaculis leo. Nullam vel metus in nunc fringilla tristique vel ut tortor.
\\\\
\noindent
Palavras-chave: latex. abntex. editoração de texto.

\end{resumo}
    %!TEX root = tcc.tex

\begin{resumo}[Abstract]
\begin{otherlanguage*}{english}

Fusce tempor nec est eu tincidunt. Aenean consectetur accumsan quam non aliquam. Vestibulum ac hendrerit massa. Aliquam dictum lacinia ullamcorper. Phasellus posuere nunc nec felis vulputate, non facilisis metus aliquet. Aenean faucibus, lacus et congue tristique, mi turpis faucibus mi, sed rhoncus nibh erat sed lorem. Vivamus erat nunc, eleifend id lorem in, euismod tempus sem.
\\\\
\noindent
Key-words: latex. abntex. text editoration.

\end{otherlanguage*}
\end{resumo}
    \listoffigures*
    \cleardoublepage
    \listoftables*
    \cleardoublepage
    %!TEX root = tcc.tex

\newdualentry[][type=abbrev]{unisul}{UNISUL}{Universidade do Sul de Santa Catarina}{Fundação criada pelo poder municipal de Tubaração-SC, em operação do sul ao norte do estado de Santa Catarina}
    %!TEX root = tcc.tex

\newdualentry{sc}{SC}{Santa Catarina}{Uma das 27 unidades federativas do Brasil, localizada no centro da região Sul do país}
    %!TEX root = tcc.tex

\newglossaryentry{sum}{
  type=symbols,
  name={Somatória},
  text={somatória},
  symbol={\ensuremath{\sum}},
  description={- \emph{Summation} - Soma de todos os valores no invervalo}}
    \tableofcontents*
    \cleardoublepage

    % Elementos textuais
    \textual
    \chapter{Cap. 1}

\lipsum
    \chapter{Cap. 2}

\lipsum

\begin{figure}[h]
\centering
\caption{Logo da Unisul}
\includegraphics[width=90mm]{logo_unisul}
\caption*{Fonte: http://unisul.br/}
\end{figure}

\lipsum[1]
    \chapter{Cap. 3}

\lipsum
    \chapter{Cap. 4}

Exemplo de citação longa:

\begin{citacao}
\lipsum[1] \cite[5.3]{abntex2cite-alf}
\end{citacao}

Exemplo de citação curta:

Segundo \cite{abntex2cite}, "Lorem ipsum dolor sit amet, consectetur adipiscing elit. Suspendisse adipiscing eu libero non ultricies. Donec accumsan, turpis ut malesuada facilisis, felis augue porta quam, et lacinia turpis ligula et risus.".
    \chapter{Cap. 4}

\lipsum

    % Elementos pós-textuais
    \postextual
    %!TEX root = tcc.tex

\newglossaryentry{dissertacao}{
  name=Dissertação,
  text=dissertação,
  description={Uma modalidade de redação ou composição, escrita em prosa ou apresentada de forma oral}}

    \apendices

\chapter{Primeiro apêndice}
Primeiro apêndice!

\chapter{Segundo apêndice}
Segundo apêndice
    %!TEX root = tcc.tex

\anexos

\chapter{Primeiro anexo}
Primeiro anexo!

\chapter{Segundo anexo}
Segundo anexo!
    \bibliography{tcc}

\end{document}